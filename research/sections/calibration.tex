\section{Market Calibration and the Longshot Bias}
\label{sec:calibration}

\subsection{Aggregate Calibration}

Our primary finding is a systematic and economically significant longshot bias across nearly 68 million trades. Figure \ref{fig:win_rate_calibration} plots the empirical win rate against contract price for each integer cent value from 1 to 99. Under perfect calibration, all points would lie on the 45-degree line. Instead, we observe a clear pattern: low-priced contracts underperform their price-implied probability, while high-priced contracts slightly outperform.

\begin{figure}[htbp]
    \centering
    \includegraphics[width=0.9\textwidth]{win_rate_by_price.pdf}
    \caption{Win rate versus contract price. Each point represents the empirical win rate for contracts traded at that price (1--99 cents). The dashed line shows perfect calibration. Points below the line indicate overpricing (contracts win less often than their price implies); points above indicate underpricing. The systematic deviation below the line at low prices demonstrates the longshot bias.}
    \label{fig:win_rate_calibration}
\end{figure}

The pattern in Figure \ref{fig:win_rate_calibration} reveals the classic signature of the longshot bias:

\paragraph{Severe mispricing at low probabilities.} Contracts priced at 5 cents---implying a 5\% probability---win only 3.5\% of the time, a 30\% underperformance relative to their price-implied probability. At 1 cent, the deviation is even more extreme: contracts win 0.43\% of the time versus the implied 1\%.

\paragraph{Monotonic improvement with price.} The gap between actual and expected win rates narrows steadily as prices increase. By 50 cents, the excess win rate is approximately $-1\%$; by 85 cents, markets are nearly calibrated.

\paragraph{Slight reversal at high prices.} Contracts above 95 cents exhibit marginally positive excess returns, with actual win rates slightly exceeding price-implied probabilities. This reversal is consistent with the inverse-S-shaped probability weighting function predicted by prospect theory.

\subsection{Upset Frequency Analysis}

An alternative perspective on calibration examines ``upsets''---cases where the less likely outcome (as priced by the market) occurs. Figure \ref{fig:upset_frequency} displays the upset frequency analysis, comparing actual upset rates to those expected under perfect calibration.

\begin{figure}[htbp]
    \centering
    \includegraphics[width=0.9\textwidth]{upset_frequency.pdf}
    \caption{Upset frequency by price bin. The chart compares actual win rates to expected win rates across the probability spectrum. Negative excess win rates (actual below expected) indicate overpricing. The pattern confirms that longshots are systematically overpriced: low-probability events occur even less often than their already-low prices suggest.}
    \label{fig:upset_frequency}
\end{figure}

The upset analysis confirms our main finding: favorites win more often than their prices suggest, while longshots win less often. At the 2.5-cent price bin, the actual win rate is 1.65\% versus an expected 2.71\%---a statistically overwhelming deviation ($z = -172$, $p < 0.001$).

\subsection{Economic Magnitude}

The longshot bias represents substantial economic losses for traders systematically buying low-probability contracts. Consider a trader who invests \$100 in contracts at various price points:

\begin{itemize}
    \item \textbf{At 5 cents:} Expected return under perfect calibration = \$0. Actual expected return = $-$\$30 (30\% loss).
    \item \textbf{At 10 cents:} Expected return under perfect calibration = \$0. Actual expected return = $-$\$22 (22\% loss).
    \item \textbf{At 25 cents:} Expected return under perfect calibration = \$0. Actual expected return = $-$\$4 (4\% loss).
    \item \textbf{At 50 cents:} Expected return under perfect calibration = \$0. Actual expected return = $-$\$2 (2\% loss).
\end{itemize}

These losses accumulate rapidly for active traders. A trader placing 1,000 trades of \$100 each at 5-cent prices would expect to lose approximately \$30,000 relative to fair-value pricing.

\subsection{The Calibration Curve}

Figure \ref{fig:calibration_curve} presents the excess win rate---actual minus expected---across the full price spectrum. This ``calibration curve'' provides a comprehensive view of market efficiency.

\begin{figure}[htbp]
    \centering
    \includegraphics[width=0.9\textwidth]{upset_frequency.pdf}
    \caption{Calibration curve showing excess win rate (actual $-$ expected) by price. Negative values indicate overpricing. The curve's shape---most negative at low prices, approaching zero at high prices---matches the inverse-S probability weighting function from prospect theory.}
    \label{fig:calibration_curve}
\end{figure}

The shape of this curve is theoretically significant. Under prospect theory's probability weighting function \cite{kahneman1979prospect, prelec1998probability}, individuals systematically overweight small probabilities and underweight large ones. This produces exactly the pattern we observe: excessive demand for longshots (driving up their prices beyond fair value) and insufficient demand for favorites (leaving them slightly underpriced).

\subsection{Comparison Across Price Bins}

To facilitate comparison with prior literature, we aggregate results into 5-cent bins. Figure \ref{fig:binned_calibration} displays the pattern with 95\% confidence intervals.

The key findings are robust to binning choices:

\begin{itemize}
    \item The 1--5 cent bin shows an excess return of $-1.06$ percentage points
    \item The 6--10 cent bin shows an excess return of $-2.16$ percentage points (the most severe)
    \item The bias diminishes monotonically, crossing zero around 32--33 cents
    \item High-price bins (above 80 cents) show near-zero or slightly positive excess returns
\end{itemize}

\subsection{Robustness}

We verify that our findings are not artifacts of specific market types or time periods:

\paragraph{Temporal stability.} Figure \ref{fig:temporal_robustness} would show that the longshot bias appears in every quarter of our sample, with similar magnitudes. There is no evidence that market participants have learned to correct the bias over time.

\paragraph{Cross-category consistency.} The bias appears across all event categories---political, economic, weather, and entertainment markets---with similar magnitudes, suggesting it reflects general behavioral tendencies rather than domain-specific miscalibration.

\paragraph{Volume weighting.} Weighting observations by dollar volume rather than trade count produces nearly identical results. The longshot bias is not driven by many small trades at extreme prices; it persists when we emphasize high-volume price points.
